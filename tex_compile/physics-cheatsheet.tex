\documentclass[12pt,a4paper]{article}

% packages
\usepackage[utf8]{inputenc}
\usepackage[ngerman]{babel}
\usepackage[T1]{fontenc}
\usepackage{amsmath}
\usepackage{amsfonts}
\usepackage{amssymb}
\usepackage{amsopn}
\usepackage{dsfont}
\usepackage[hidelinks]{hyperref}
\usepackage{graphicx}
\usepackage[dvipsnames]{xcolor}
\usepackage[margin=2.5cm,left=3cm]{geometry}
\usepackage{setspace}
\usepackage[defaultlines=3,all]{nowidow}
\usepackage{physics}
\usepackage[many]{tcolorbox}
\usepackage{chemfig}

% config
%\setlength{\parindent}{0pt} % indent first line
\pagestyle{plain}
\onehalfspacing


% commands
% % access theorems
\newcommand{\Def}[2][]{
	\noindent
	\newline
	\vspace*{0.3\baselineskip}
	\textbf{Definition:} #1
	\newline
	#2
	\vspace*{\baselineskip}
}

% % equation arrays
\newcommand{\aligns}[1]{\begin{align*}#1\end{align*}}
\newcommand{\eqn}[1]{\begin{eqnarray*}#1\end{eqnarray*}}
\newcommand{\eqnno}[1]{\begin{eqnarray}#1\end{eqnarray}}

% % boxes
\newcommand{\eqnbox}[1]{
	\begin{center}
		\begin{tcolorbox}[width=0.9\textwidth]
			\abovedisplayskip=-3pt
			\begin{eqnarray}
				#1
			\end{eqnarray}
		\end{tcolorbox}
	\end{center}
}
\newcommand{\tbox}[1]{
	\begin{center}
		\begin{tcolorbox}
			#1
		\end{tcolorbox}
	\end{center}
}

% % matrices
\newcommand{\IdMatrix}{\mathds{1}}
\newcommand{\Transposed}[1]{#1^{\,\mathsf{T}}}
\newcommand{\tp}[1]{\Transposed{#1}}
\newcommand{\Mspace}{\:\:} % fuer Einstein-Summenkonvention (Abstand zum rechten Index)
\newcommand{\Cases}[1]{\begin{cases}#1\end{cases}}

% % % colors
% % basic colors
\newcommand{\Red}[1]{\textcolor{Red}{#1}}
\newcommand{\Blue}[1]{\textcolor{Blue}{#1}}
% % bold printed colors
\newcommand{\Redbf}[1]{\textbf{\Red{#1}}}
\newcommand{\Bluebf}[1]{\textbf{\Blue{#1}}}

% set custom references
\makeatletter
\newcommand{\manuallabel}[2]{\def\@currentlabel{#2}\label{#1}}
\makeatother
% usage: 
%	set label to sect. 2
%	\label{key}
% \manuallabel{label}{key}

% mass
\DeclareMathOperator{\KiloGramm}{kg}
\DeclareMathOperator{\Gramm}{g}
\DeclareMathOperator{\mGramm}{mg}
\DeclareMathOperator{\mmGramm}{\mu g}
% length
\DeclareMathOperator{\KiloMeter}{km}
\DeclareMathOperator{\Meter}{m}
\DeclareMathOperator{\cMeter}{cm}
\DeclareMathOperator{\mmMeter}{mm}
% time
\DeclareMathOperator{\Second}{s}
\DeclareMathOperator{\mSecond}{ms}
\DeclareMathOperator{\mmSecond}{\mu s}
% volume
\DeclareMathOperator{\Liter}{l}
% other (physical)
\DeclareMathOperator{\Newton}{N}


% math operators
\DeclareMathOperator{\const}{const}

% authors info
\title{title} 
\author{me}
\date{\today}

\author{me}
\title{
	\LaTeX\\
	\textsc{physics}--Cheatsheet\\
	\small{
		Zusammenfassende Übersicht über das \textsc{physics}--Paket\footnote{Quelle und ausführliche Dokumentation:
			\url{https://www.ctan.org/pkg/physics}
		}
	}
}
\date{\today}

\begin{document}
\maketitle
\tableofcontents

\clearpage
\section{Automatische Klammersetzung}
\aligns{
	% nur klammern
	\qty(1) &&				% klammern, rund
	\qty[1] &&				% klammern, eckig
	\qty|1| &&				% klammern, gerade
	\qty{1}\\				% klammern, geschweift
	% Groesse manuell veraendern / anpassen:
	% automatisch - big - Big - bigg - Bigg
	\qty(\frac{1}{2}) &&
	\qty\big(\frac{1}{2})&&
	\qty\Big(\frac{1}{2})&&
	\qty\bigg(\frac{1}{2})&&
	\qty\Bigg(\frac{1}{2})\\
	% anderes
	\\\\
	\abs{1} &&				% Betrag
	\norm{1} \\				% Norm
	\eval{x}_0^\infty&& 	% Evaluierung
	\order{x^2} \\			% Ordnung x^2
	&&
	\comm{A}{B}  &&			% Kommutator
	\acomm{A}{B} && 		% Antikommutator
	\pb{A}{B}				% Poisson-Klammern	
}

\section{Vektornotationen}
\eqn{
	% mit *: zzgl. italic / greek
	\vb{a} 	&\rightarrow& \vb*{a}\\				% vector bold, no greek (gets bold)
	\vb{\theta} &\rightarrow& \vb*{\theta}\\	% vector bold italic, greek gets bold
	\vu{a}\\										% vector unit
	\vb{a} \vdot \vb{b}\\							% dot product
	\vb{a} \cp \vb{b}\\								% cross product
	\grad &:&										% gradient
	\grad{\Psi}\qc
	\grad(\Psi+\Phi)\qc									% long-form (wie mit qty)
	\grad[\Psi+\Phi]\\									% long-form (wie mit qty)
	\div &:&										% divergence
	\div{\vb{a}}\qc
	\div(\Psi+\Phi)\qc									% long-form (wie mit qty)
	\div[\Psi+\Phi]\\									% long-form (wie mit qty)
	\curl &:&										% rotation
	\curl{\vb{a}}\qc
	\curl(\vb{a}+\vb{b})\qc
	\curl[\vb{a}+\vb{b}]\\
	\laplacian &:&									% laplacian		
	\laplacian{\vb{a}}\qc
	\laplacian(\vb{a}+\vb{b})\qc
	\laplacian[\vb{a}+\vb{b}]\\
}


\section{Operatoren}
Sinus etc. brauchen \emph{kein} \textbackslash qty für angepasste Klammern!
\eqn{
	\sin(x)\qc		% normale Form
	\sin[2](x)\qc	% optional: power (sin^x)
	\sin x			% funktioniert auch ohne Argument
}
Analog funktionieren:
\aligns{
	\sin(x) &&
	\cos(x) &&
	\tan(x) &&
	\csc(x) &&
	\sec(x) &&
	\cot(x)\\
	\sinh(x) &&
	\cosh(x) &&
	\tanh(x) &&
	\csch(x) &&
	\sech(x) &&
	\coth(x)\\
	\arcsin(x) &&
	\arccos(x) &&
	\arctan(x) &&
	\arccsc(x) &&
	\arcsec(x) &&
	\arccot(x)\\
	\exp(x) &&					% exponential
	\log(x) &&					% logarithm
	\ln(x) &&					% naturallogarithm
	\det(x) &&					% determinant
	\Pr(x)	&&					% propability
	\tr M \\					% Spur (trace), Klammern wie \qty
	\Tr M &&					% trace
	\rank M &&					% Rank 
	\erf M &&					% gaussian error function
	\Res[f(z)]	\\				% residue, Klammern wie \qty
	\Re{z} &&					% Realteil
	\real(z) &&
	\Im{z} &&					% Imaginärteil
	\imaginary(z) &&
%		\pv{\int f(z) \dd{z}} &&	% cauchy principal value
%		\PV{\int f(z) \dd{z}}\\		% cauchy principal value
}

\section{Texte mit Abständen einfügen}
% quick quad text
\aligns{
	1 \qq{, text}  2	&&							% analog zu \quad\text{}\quad
	1 \qq*{, text} 2	&&							% analog zu \text{}\quad
	\qc \\											% \qcomma
}
geht analog auch für:
\emph{then, else, otherwise, unless, given, using, assume, since, let, for, all, even, odd, integer, and, or, as, in}

\section{Ableitungen}
\eqn{
	\dd &:&						% differenzial, das normale "d"
	\dd x\qc					% "dx"
	\dd{x}\qc					% "	dx "
	\dd[3]{x}\qc				% "d^3x"
	\dd(\cos\theta)				% "d(\cos\theta)
	\\\\
	\dv{x} &:&					% d/dx (Bruchschreibweise)
	\dv{f}{x}\qc				% df/dx (Bruchschreibweise)
	\dv[n]{f}{x}\qc
	\dv{x}(\dv{L}{\dot{x}})\qc	% mit Klammern wie \qty
	\dv*{}{t}					% wie "df/dx" in einer Zeile, nutzt \flatfrac
	\\\\
	\pdv{x} &:&					% partielle Ableitung als Bruch (hier, in den Kommentaren, schreibe ich "D" statt \partial)
	\pdv{f}{x}\qc				% Df/Dx
	\pdv[n]{f}{x}\qc			% D^nf/Dx^n
	\pdv{f}{x}{y}\\				% gemischte Schreibweise: Df/DxDy
%		\pdv{f}{x}{y}{z}			% !!!geht nicht mehr!!!
}
Sonstige Ableitungen (für uns Physiker unüblich):
\eqn{
	% funktioniert wie \dd
	\var{F[g(x)]} &&			% Variation: F[g(x)] 
	\var(E-TS)\\					% mit Klammern wie \qty
	% funktioniert wie \dv
	\fdv{g} &:&					% Funktionalableitung \delta / \delta g
	\fdv{F}{g}\qc						% DF/Dg
	\fdv{V}(E-TS)\qc					% langform
	\fdv*{F}{x}						% mit \flatfrac
}

\section{Dirac bra--ket Notation}
Um die Klammer $\bra{\phi}\ket{\psi}$ korrekt darzustellen, darf kein Abstand zwischen den beiden Klammern sein. Ansonsten sieht es wie in diesem Satz $\bra{\phi} \ket{\psi}$ aus. Zudem gibt es noch mehrere andere Funktionen:
\eqn{
	\braket{a}{b} \qc \braket{a} \qc \ip{a}{b}  \\		% braket
	\ketbra{a}{b} \qc \op{a}{b} \qc	\op{a}\\	  		% ketbra
	\expval{A} \qc \expval{A}{\Psi} \qc \ev{A}{\Psi} \\	% expectationvalue
	% ohne * wird der mittlere Block von der Größenanpassung ignoriert
	% mit  * wird keine Größenanpassung vorgenommen
	% mit ** wird der mittlere Block für die Größenanpassung berücksichtigt
	\ev{\text{\large{A}}}{\Psi} \qc \ev*{\text{\large{A}}}{\Psi} \qc \ev**{\text{\large{A}}}{\Psi}\\ 
	\matrixel{n}{A}{m} \qc \mel{n}{A}{m}				% matrixelement, all 3 elements required
}

\section{Matrizen}
\eqn{
	\mqty(\imat{2}) &&\\ 		% identity matrix in n dimensions
	\mqty(\imat{2}) \qquad \mqty*(\imat{2}) && \mqty[\imat{2}] \qquad \mqty{\imat{2}} \qquad \mqty|\imat{2}| \\	% Matrizen-Umgebung
	\mdet{\imat{2}} \\ 			% matrix determinant
	\imat{2}  		\\			% just the ELEMENTS of the identity matrix
	% xmat: elements for n-m-Matrix same element x
	\mqty(\xmat{x}{2}{4}) && \mqty(\xmat*{x}{2}{4}) \\		% xmat* adds element indices
	\mqty(\zmat{2}{5})	\\		% elements of zero matrix
	\mqty(\pmat{0}) \qquad \mqty(\pmat{1}) && \mqty(\pmat{2}) \qquad \mqty(\pmat{3}) \\ % pauli matrix
	\mqty(\dmat{1,2,3}) \qquad \mqty(\dmat[0]{1,2}) && \mqty(\dmat{1,2&3\\4&5}) \\		% diagonal matrix
	\mqty(\admat{1,2,3})		% anti-diagonal matrix
}
Zudem gibt es z.\,B. $\smqty(\imat{2})$, was eine verkleinerte Darstellung mittels \emph{smallmatrix} erzeugt.
\end{document}