% % access theorems
\newcommand{\Def}[2][]{
	\noindent
	\newline
	\vspace*{0.3\baselineskip}
	\textbf{Definition:} #1
	\newline
	#2
	\vspace*{\baselineskip}
}

% % equation arrays
\newcommand{\aligns}[1]{\begin{align*}#1\end{align*}}
\newcommand{\eqn}[1]{\begin{eqnarray*}#1\end{eqnarray*}}
\newcommand{\eqnno}[1]{\begin{eqnarray}#1\end{eqnarray}}

% % boxes
\newcommand{\eqnbox}[1]{
	\begin{center}
		\begin{tcolorbox}[width=0.9\textwidth]
			\abovedisplayskip=-3pt
			\begin{eqnarray}
				#1
			\end{eqnarray}
		\end{tcolorbox}
	\end{center}
}
\newcommand{\tbox}[1]{
	\begin{center}
		\begin{tcolorbox}
			#1
		\end{tcolorbox}
	\end{center}
}

% % matrices
\newcommand{\IdMatrix}{\mathds{1}}
\newcommand{\Transposed}[1]{#1^{\,\mathsf{T}}}
\newcommand{\tp}[1]{\Transposed{#1}}
\newcommand{\Mspace}{\:\:} % fuer Einstein-Summenkonvention (Abstand zum rechten Index)
\newcommand{\Cases}[1]{\begin{cases}#1\end{cases}}

% % % colors
% % basic colors
\newcommand{\Red}[1]{\textcolor{Red}{#1}}
\newcommand{\Blue}[1]{\textcolor{Blue}{#1}}
% % bold printed colors
\newcommand{\Redbf}[1]{\textbf{\Red{#1}}}
\newcommand{\Bluebf}[1]{\textbf{\Blue{#1}}}

% set custom references
\makeatletter
\newcommand{\manuallabel}[2]{\def\@currentlabel{#2}\label{#1}}
\makeatother
% usage: 
%	set label to sect. 2
%	\label{key}
% \manuallabel{label}{key}
